\section{Prise en charge des utilisateurs et développeurs}

Selon la figure \ref{fig:gitorg},  Les développeurs communiquent avec le \serv{} via le réseau. On utilisé 
Le protocol \lstinline{ssh}. Chaque développeur doit avoir un compte sur le Mac \serv{}., qui contient les \lstinline{repositories git bare}.  Il doit aussi avoir une compte \lstinline{git}. Il y aura un répertoire \lstinline{/Users/git/reposGit} pour contenir les repositories.

\subsection{Création de l'utilisateur \lstinline{git} sur le serveur}
Aller Apple Menu  \ra{} System settings \ldots \ra{} Users \& Groups \ra{} \userX{} bouton Add User.

Nom complet \lstinline{guiU // sans importance}

Nom du compte (shortname important !) : \lstinline{git}

Donner mot de passe

Créer l'utilisateur.

Se loger sur l'utilisateur \lstinline{git} et créer le répertoire \lstinline{reposGit}. 
Doner \lstinline{mkdir} et le nom \lstinline{reposGit}. 

\subsection{Créer un groupe \lstinline{gitgroup}}

Aller Apple Menu  \ra{} System settings \ldots \ra{} Users \& Groups \ra{} \userX{} bouton Add Group \lstinline{gitgroup}.

Ensuite donner :

\begin{lstlisting}
sudo chown git /Users/git/reposGit
sudo chgrp -R gitgroup /User/git/reposGit
\end{lstlisting}

Vérifier :

\begin{lstlisting}
ls -ld /Users/git/repoGit
\end{lstlisting}

Résultat attendu :
\begin{lstlisting}
drwxrwsr-x git gitgroup ...  /Users/git/reposGit
\end{lstlisting}

\subsection{Création des utilisateur (développeur) sur le serveur}
Il faut créer des comptes utilisateurs sur le serveur macOS (\serv{}) via l’interface graphique.

Aller Apple Menu  \ra{} System settings \ldots \ra{} Users \& Groups \ra{} \userX{} bouton Add User.

Choisir : Utilisateur standard (recommandé pour accès \lstinline{SSH} + \git{})

Donner :

Nom complet \lstinline{r g r  // sans importance}

Nom du compte (shortname important !), par exemple : \userOne{}

Mot de passe

Créer l'utilisateurs nécessaires.

Aller Apple Menu  \ra{} System settings \ldots \ra{} Users \& Groups \ra{} \userX{} bouton I de  \lstinline{gitgroup} et ajouter les utilisateurs qui vont utiliser \git{}.

\subsection{Vérification sur le serveur que les comptes existent}

\begin{lstlisting}
dscl . -list /Users
\end{lstlisting}


\subsection{Activer "Remote Login" sur serveur}

Pour avoir un accès par un client à l'extérieur du serveur. Sur le serveur configurer:

Aller Apple Menu  \ra{} System settings \ldots \ra{} General \ra{} Sharing Activer Remote Login.

Activer File Sharing, Remote Management, RemoteLoging, et Remote Application Scripting.
À chaque option aller dans i et faites le nécessaire.


\subsection{Vérification  du remote login sur serveur}

Dans le terminal du serveur :

\begin{lstlisting}
sudo systemsetup -getremotelogin
\end{lstlisting}

Doit afficher :

\begin{lstlisting}
Remote Login: On
\end{lstlisting}



\subsection{Vérification qu'un utilisateur sur Mac client  a accès au serveur}
Accéder au serveur depuis un autre Mac
Connexion SSH simple (mot de passe)

Sur le Mac user :

\begin{lstlisting}
ssh rgrmini1@aplixserveur.local
\end{lstlisting}



La première fois, il faut répondre :

\begin{lstlisting}
Are you sure you want to continue connecting (yes/no/[fingerprint])?
\end{lstlisting}

Donner 

\begin{lstlisting}
yes
\end{lstlisting}
