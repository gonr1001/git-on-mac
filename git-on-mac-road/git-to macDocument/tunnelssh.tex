\section{Le tunnel ssh}
Un tunnel SSH (ou SSH port forwarding) sert à faire passer du trafic réseau à travers une connexion SSH chiffrée.

Il permet :

\begin{itemize}

\item Sécuriser un service non sécurisé

Exemple : accéder en HTTPS via un tunnel à un serveur HTTP local non exposé.

\item Accéder à un service distant comme s’il était local

Exemple : accéder à un serveur SVN, MySQL, ou une interface web située derrière un firewall.

\item Contourner des restrictions réseau

Exemple : accéder à une machine du LAN privé via un serveur bastion.

\end{itemize}

Le principe :

Le client SSH ouvre un port local (ex. : 8080)

Tout ce qui va à localhost:8080 passe dans un tunnel chiffré

Le serveur SSH renvoie le trafic à la machine/port cible

Sécuriser SVN

Le protocole svn:// n’est pas chiffré.
Avec SSH → tout est chiffré automatiquement.

 Accéder à un serveur SVN non exposé

Par exemple, si le serveur SVN écoute en interne sur localhost:3690, sans être ouvert sur Internet.

Contourner un firewall / routeur

Si l’accès direct au port SVN (3690) est bloqué.

 Accéder au repository SVN comme s’il était en local

Le client SVN croit qu’il parle à localhost, mais le tunnel redirige vers le serveur.

\subsection{ Configuration du tunnel SSH pour SVN}

Supposons :

Serveur SVN écoute sur : localhost:3690

Serveur SSH : user@serveur.example.com

Sur ton Mac, on crée un port local : 3690

Commande du tunnel SSH (mode Local Forwarding)
\begin{lstlisting}
ssh -L 3690:localhost:3690 user@serveur.example.com    
\end{lstlisting}



Ce que cela fait :

Quand SVN accède à localhost:3690 sur ton Mac
→ le trafic passe par SSH
→ et arrive sur serveur.example.com:3690

\subsection{Vérification du tunnel SSH avant d’utiliser SVN} 
Étape 1 : Vérifier que le tunnel fonctionne

Dans un autre terminal, après avoir lancé le tunnel :

nc -zv localhost 3690


Tu dois voir :

Connection to localhost port 3690 succeeded!

Étape 2 : Vérifier que SVN répond depuis le tunnel

Toujours dans un terminal :

svn info svn://localhost/


Si cela fonctionne → tu as bien accès au repo via SSH.

\subsection{Utiliser SVN via le tunnel SSH}

Dans ton client SVN, tu peux maintenant utiliser :

\begin{lstlisting}
 svn co svn://localhost/nom_du_repo   
\end{lstlisting}


ou

svn ls svn://localhost/


 Le client parle à localhost, mais en réalité il utilise le port folwardé via SSH.

\subsection{Tunnel SSH persistant pour SVN (facultatif)}

Pour éviter de relancer la commande manuellement :

Installer autossh (macOS) :
brew install autossh

Lancer un tunnel fiable :
autossh -M 0 -L 3690:localhost:3690 user@serveur.example.com

\subsection{Problèmes fréquents (et solutions)}
 Le tunnel ne se connecte pas

Vérifier le port SSH :

ssh user@serveur.example.com

 SVN renvoie une erreur « Connection refused »

Le service SVN n’écoute peut-être que sur IPv4.
Essaye :

ssh -L 3690:127.0.0.1:3690 user@serveur.example.com

 Le serveur SVN n’est pas lancé

Sur le serveur :

\begin{lstlisting}
 ps aux | grep svnserve   
\end{lstlisting}



ou lancer :

svnserve -d -r /chemin/vers/repos

\subsection{Résumé rapide}
Étape	Action
1	Lancer tunnel SSH : ssh -L 3690:localhost:3690 user@serveur
2	Vérifier port : nc -zv localhost 3690
3	Vérifier SVN : svn info svn://localhost/
4	Utiliser SVN : svn co svn://localhost/repo