\documentclass{article}

\usepackage[T1]{fontenc} 
%\RequirePackage{utf8}{inputenc}
%%%%%%% à
%%%%%%% located at forAll

\usepackage[utf8]{inputenc}



\renewcommand{\rmdefault}{ptm}
\renewcommand{\bfdefault}{b}
\usepackage[scaled=0.92]{helvet}
\usepackage{courier}
\normalfont % in case the EC fonts aren't available

%\usepackage[T1]{fontenc}
%\RequirePackage[latin1]{inputenc}


\NeedsTeXFormat{LaTeX2e}
%\usepackage{applekeys}
%\usepackage{titling}
%\usepackage{titlesec}
\usepackage{latexsym}
\RequirePackage[french]{babel}    % in class
\RequirePackage{algorithm}
\RequirePackage{algorithmic}
\usepackage{listings} %  in class
\RequirePackage{url}  % in class
\RequirePackage{xcolor} %  in class

\usepackage{tikz}
\usetikzlibrary{automata,positioning,arrows}

\def\eXit{$\epsilon$\kern-.100em \lower.5ex\hbox{X}\kern-.125em it}

%\usepackage{tikz-uml}

\usepackage{fancyvrb}
\usepackage{verbatim}

\usepackage{graphicx}

\usepackage{epigraph} 
\epigraphnoindent

\newcommand{\pasImp}{\color{red}Pas à implémenter.\color{black}}

\bibliographystyle{alpha}
\usepackage{listings}

\definecolor{pblue}{rgb}{0.13,0.13,1}
\definecolor{pgreen}{RGB}{0,0.5,0}
\definecolor{pred}{rgb}{0.9,0,0}
\definecolor{pgrey}{rgb}{0.46,0.45,0.48}
\definecolor{vertUdeS}{RGB}{0, 128, 85}


\usepackage{inconsolata}
\usepackage{textcomp}

\usepackage{longtable}
\usepackage{lipsum} % just for dummy text- not needed for a longtable

\usepackage{hyperref}

\RequirePackage{url}
\RequirePackage[letterpaper,includeheadfoot,top=.5in,bottom=.5in,bindingoffset=.5in]{geometry}  %------ Format de page

\lstset{language=Java,
	basicstyle=\linespread{0.8}]\ttfamily,
  showspaces=false,
  showtabs=false,
  breaklines=true,
  tabsize=2,
  showstringspaces=false,
  breakatwhitespace=true,
  commentstyle=\color{pgreen},
  keywordstyle=\color{pblue},
  stringstyle=\color{pred},
  basicstyle=\ttfamily,
  moredelim=[il][\textcolor{pgrey}]{$$},  % this is not an error
  moredelim=[is][\textcolor{pgrey}]{\%\%}{\%\%}
}



\def\hdrule{\hrule \vskip2pt \hrule}
\def\mots{\underline{\hspace*{0.75in}}\ }

\def\eXit{$\epsilon$\kern-.100em \lower.5ex\hbox{X}\kern-.125em
it}


\newcommand{\brun}{\noindent $\triangleright$}
\newcommand{\erun}{$\triangleleft$}
\newcommand{\key}{\textsf}
\newcommand{\ita}{\textit}
\newcommand{\dos}{\textsc}
\newcommand{\bld}{\textbf}
\newcommand{\ang}{\textsf}
\newcommand{\pro}{\texttt}

%/newcommand{\name}{\textit{draw4US}}

\newcommand{\ints}{\renewcommand{\baselinestretch}{1.0}\small \normalsize}
\newcommand{\intm}{\renewcommand{\baselinestretch}{1.5}\small \normalsize}
\newcommand{\intd}{\renewcommand{\baselinestretch}{2.0}\small \normalsize}

\newcommand{\rbl}{\rule[-1mm]{10mm}{4mm}}

\newcommand{\diamant}{DIAMANT}
\newcommand{\dBug}{DiamantBug}
\newcommand{\dWeb}{DiamantWeb}
\newcommand{\dTri}{DiamantTri}
\newcommand{\jreplay}{jReplay}
\newcommand{\saphir}{SAPHIR}

\newcommand{\applic}{\emph{EMR}}
\newcommand{\XP}{\emph{eXtreme Programming}}
\newcommand{\oop}{\og \emph{object-oriented programming} \fg{}}
\newcommand{\refac}{\og \emph{refactorning} \fg{}}
\newcommand{\dcloope}{d'ouverture-fermeture{}}

\newcommand{\cloope}{ouverture-fermeture{}}
\newcommand{\design}{\og \emph{design} \fg{}}
\newcommand{\DesPats}{\og \emph{Design patterns} \fg{}}

\newcommand{\mdc}{\emph{modèle de conception}}
\newcommand{\mdcs}{\emph{modèles de conception}}
\newcommand{\Mdcs}{\emph{Modèles de conception}}

\newcommand{\Reusin}{\emph{Réusinage}}
\newcommand{\reusin}{\emph{réusinage}}
\newcommand{\reusins}{\emph{réusinages}}

\newcommand{\cImp}{\og \emph{codeImpossible} \fg{}}

\newcommand{\csessioni}{\emph{s3I}}

\newcommand{\btrack}{ \emph{retour en arrière}}

\newcommand{\steve}{Tim}

%%%%%%



\newcommand{\tdd}{\emph{développement piloté par les tests}}


\newcommand{\Scrum}{\textsl{Scrum}}
\newcommand{\ScrumTeam}{\textsl{Scrum team}}

\newcommand{\scrumMaster}{\textsl{scrum master}}

\newcommand{\productOwner}{\textsl{product owner}}

\newcommand{\developer}{\textsl{developer}}


\newcommand{\sprint}{\textsl{sprint}}
\newcommand{\sprints}{\textsl{sprints}}

\newcommand{\sprintReview}{\textsl{sprint review}}

\newcommand{\sprintRetrospective}{\textsl{sprint retrospective}}

\newcommand{\productBacklog}{\textsl{product backlog}}

\newcommand{\sprintBacklog}{\textsl{sprint backlog}}

\newcommand{\sprintPlanningMeeting}{\textsl{sprint planning meeting}}

\newcommand{\dailySprintMeeting}{\textsl{daily sprint meeting}}

\newcommand{\sprintBurndownChart}{\textsl{sprint burndown chart}}

\newcommand{\releaseBurndownChart}{\textsl{release burndown chart}}

\newcommand{\pprog}{\textsl{pair programming}}


\newcommand{\useCase}{\textsl{use case}}
\newcommand{\useCases}{\textsl{use cases}}

\newcommand{\timeBoxed}{\textsl{time boxed}}

\newcommand{\timeBox}{\textsl{time box}}


\newcommand{\release}{\textsl{release}}
\newcommand{\releases}{\textsl{releases}}

\newcommand{\artifact}{\textsl{artifact}}
\newcommand{\artifacts}{\textsl{artifacts}}

\newcommand{\uds}{Universit\'{e} de Sherbrooke}

\newcommand{\ra}{$\rightarrow$}

\def\acrobat{\hyperdef{Sauter}{ACetEndroit}{}}

\newcommand{\siteClassUML} {\href{https://fr.wikipedia.org/wiki/Diagramme_de_classes}{\texttt{https://fr.wikipedia.org/wiki/Diagramme\_de\_classes}}}

\newcommand{\siteClassUMLEng}{ \href{https://en.wikipedia.org/wiki/Class_diagram}{\texttt{https://en.wikipedia.org/wiki/Class\_diagram}}}

\newcommand{\siteAPP}{ \href{https://www.gel.usherbrooke.ca/s3i/e20/}{\texttt{https://www.gel.usherbrooke.ca/s3i/e20/}}}

\newcommand{\siteForDis} {\href{http://www.usherbrooke.ca/genie/fc/fad/info/}{\texttt{http://www.usherbrooke.ca/genie/fc/fad/info/}}}

\newcommand{\siteMoodle} {\href{http://www.usherbrooke.ca/webex/formation/capsules-dautoformation/training-center/}{\texttt{http://www.usherbrooke.ca/webex/formation/capsules-dautoformation/training-center/}}}

\newcommand{\siteGEIpAgile} {\href{http://www.usherbrooke.ca/moodle2-cours/course/view.php?id=7764}{\texttt{http://www.usherbrooke.ca/moodle2-cours/course/}}}


\newcommand{\siteCodeMakery} {\href{https://code.makery.ch/library/javafxt-utorial/}{\texttt{https://code.makery.ch/library/javafx-tutorial/}}}
\newcommand{\sitejavaFXColl} {\href{https://docs.oracle.com/javase/8/javafx/collections-tutorial/collections.htm}{\texttt{https://docs.oracle.com/javase/8/javafx/collections-tutorial/collections.htm}}}




\newcommand{\sitesvn} {\href{https://subversion.apache.org/}{\texttt{https://subversion.apache.org/}}}


\newcommand{\sitesPropBind} {\href{https://docs.oracle.com/javase/8/javafx/properties-binding-tutorial/binding.htm/}{\texttt{https://docs.oracle.com/javase/8/javafx/properties-binding-tutorial/binding.htm}}}


\newcommand{\rem}{\textbf{EMR}}
\newcommand{\application}{\textbf{EMR Application}}
\newcommand{\paintUS}{\textbf{xPaintUS}}
\newcommand{\bSmells}{\og \emph{bad smells} \fg{}}
\newcommand{\bSmell}{\og \emph{bad smell} \fg{}}


\newcommand{\mockito}{\emph{Mockito}}


\newcommand{\siteEMR} {\href{http://www.emrwebsite.org/uploads/Fichiers/Libraries/memo-EMR-2013.pdf}{\texttt{http://www.emrwebsite.org/uploads/Fichiers/memo-EMR-2013.pdf}}}


\newcommand{\sitewThreeS} {\href{http://w3sdesign.com/index0100.php}{\texttt{http://w3sdesign.com/index0100.php}}}

\newcommand{\sitetutorialsPoint} {\href{https://www.tutorialspoint.com/design\_pattern/index.htm}{\texttt{https://www.tutorialspoint.com/design\_pattern/index.htm}}}

\newcommand{\siteJavaTPoint} {\href{https://www.javatpoint.com/design-patterns-in-java}{\texttt{https://www.javatpoint.com/design-patterns-in-java}}}



\newcommand{\siteRef} {\href{http://www.refactoring.com/catalog/index.html}{\texttt{http://www.refactoring.com/catalog/index.html}}}
\newcommand{\siteArtMOO} {\href{http://www.pitt.edu/~ckemerer/CK\%20research\%20papers/MetricForOOD\_ChidamberKemerer94.pdf}{\texttt{http://www.pitt.edu/MetricForOOD\_ChidamberKemerer94.pdf}}}


\newcommand{\siteArtBug} {\href{http://thomas-zimmermann.com/publications/files/bettenburg-fse-2008.pdf}{\texttt{http://thomas-zimmermann.com/publications/files/bettenburg-fse-2008.pdf}}}



\newcommand{\siteDependency} {\href{http://rcardin.github.io/programming/oop/software-engineering/2017/04/10/dependency-dot.html}{\texttt{http://rcardin.github.io/programming/oop/dependency-dot.html}}}

\newcommand{\siteVCC} {\href{http://tictacserver.gel.usherbrooke.ca/updateVCC}{\texttt{http://tictacserver.gel.usherbrooke.ca/updateVCC}}}

\newcommand{\sitePrin} {\href{http://www.regismedina.com}{\texttt{http://www.regismedina.com}}}

\newcommand{\siteMetrics} {\href{http://metrics2.sourceforge.net/}{\texttt{http://metrics2.sourceforge.net/}}}

\newcommand{\siteDQual} {\href{https://linux.ime.usp.br/\~joaomm/mac499/arquivos/referencias/oodmetrics.pdf}{\texttt{https://linux.ime.usp.br/oodmetrics.pdf}}}

\newcommand{\siteSta} {\href{http://mil-oss.org/resources/objectmentor\_design-principles-and-design-patterns.pdf}{\texttt{http://\_design-principles-and-design-patterns.pdf}}}

\newcommand{\sitejUnit} {\href{http://junit.sourceforge.net/doc/cookstour/cookstour.htm}{\texttt{ http://junit.sourceforge.net/doc/cookstour/cookstour.htm}}}

\newcommand{\siteMockito} {\href{http://site.mockito.org/}{\texttt{http://site.mockito.org/}}}

\newcommand{\siteMockitoTut} 
{\href{http://www.vogella.com/tutorials/Mockito/article.html}{\texttt{http://www.vogella.com/tutorials/Mockito/article.html}}}


\newcommand{\siteIeee} {\href{http://www.ieee.org/publications\_standards/publications/authors/authors\_journals.html}{\texttt{http://www.ieee.org/publications/authors\_journals.html}}}

\newcommand{\siteremy} {\href{http://remy-manu.no-ip.biz/}{\texttt{http://remy-manu.no-ip.biz/}}}

\newcommand{\sitecommu}{\href{http://www.oracle.com/technetwork/java/javase/community/javafx-community-2158661.html}{\texttt{http://www.oracle.com/technetwork/java/javase/community/javafx-community-2158661.html}}}


\newcommand{\siterOracle} {\href{https://docs.oracle.com/javase/tutorial/}{\texttt{https://docs.oracle.com/javase/tutorial/}}}



\newcommand{\siterOraFx} {\href{https://docs.oracle.com/en/java/javase/11/}{\texttt{https://docs.oracle.com/en/java/javase/11/}}}

\newcommand{\siterDepen} {\href{http://tutorials.jenkov.com/ood/understanding-dependencies.html}{\texttt{http://tutorials.jenkov.com/ood/understanding-dependencies.html}}}

\newcommand{\siteTutFX} {\href{http://code.makery.ch/library/javafx-8-tutorial/part1/}{\texttt{http://code.makery.ch/library/javafx-8-tutorial/part1/}}}

\newcommand{\siteSlidesDP} {\href{https://users.encs.concordia.ca/~gregb/home/PDF/se_design_patterns.pdf}{\texttt{https://users.encs.concordia.ca/home/PDF/se\_design\_patterns.pdf}}}


%\pagestyle{udsSpecial}			
%%%%%%%%%%%%%%%%%%%%%%%%%%%%


\newcommand{\svn}{\texttt{svn}}

\newcommand{\git}{\texttt{git}}



\title{Road to git}
\author{Ruben Gonzalez-Rubio}
\date{janvier 2026}

\begin{document}

\maketitle

\tableofcontents

\listoftables
\listoffigures

\section{Introduction à git}

Git est un système de contrôle de versions distribué créé par Linus Torvalds. Il permet de suivre l’évolution des fichiers d’un projet (souvent du code source), de revenir à des versions précédentes et de travailler efficacement à plusieurs. Contrairement aux systèmes centralisés, chaque utilisateur possède une copie complète du dépôt (repository) sur sa machine, incluant l’historique complet des modifications. Cela rend Git très rapide, robuste et utilisable même sans connexion réseau.

Le fonctionnement de Git repose sur des snapshots (instantanés) plutôt que sur une simple liste de différences entre fichiers. À chaque commit, Git enregistre l’état exact du projet. Les commits sont identifiés par des hashs cryptographiques (SHA-1 ou SHA-256), ce qui garantit l’intégrité des données : toute modification non autorisée de l’historique est immédiatement détectable. Git gère trois zones principales : le working directory (fichiers modifiés), l’index (staging area) et le repository local, ce qui donne un contrôle précis sur ce qui est réellement enregistré.

Un concept central de Git est la branche (branch). Une branche représente une ligne de développement indépendante, ce qui permet de travailler sur une fonctionnalité, un correctif ou une expérimentation sans affecter le code principal (souvent la branche main). Les branches sont légères et rapides à créer. Une fois le travail terminé, elles peuvent être fusionnées (merge) ou rebasées (rebase), facilitant ainsi la collaboration et la gestion de projets complexes.

\subsection{Sécurité}

La sécurité de Git repose d’abord sur l’intégrité des données. Chaque objet Git (commit, arbre, fichier) est identifié par un hachage cryptographique calculé à partir de son contenu. Cela garantit que toute modification non autorisée de l’historique ou des fichiers est immédiatement détectable, car le hash ne correspondrait plus. Ce mécanisme protège contre la corruption accidentelle et rend très difficile la falsification silencieuse de l’historique d’un projet.

L’utilisation de SSH pour accéder aux dépôts Git distants renforce la sécurité des échanges réseau. SSH assure l’authentification forte des utilisateurs, généralement par clés cryptographiques (clé publique / clé privée), évitant l’usage de mots de passe transmis sur le réseau. Les communications sont chiffrées, ce qui protège le code source contre l’interception, l’écoute ou les attaques de type “man-in-the-middle”, même
%\input{intro}

%\section{Prise en charge des utilisateurs et développeurs}

Selon la figure \ref{fig:gitorg},  Les développeurs communiquent avec le \serv{} via le réseau. On utilisé 
Le protocol \lstinline{ssh}. Chaque développeur doit avoir un compte sur le Mac \serv{}., qui contient les \lstinline{repositories git bare}.  Il doit aussi avoir une compte \lstinline{git}. Il y aura un répertoire \lstinline{/Users/git/reposGit} pour contenir les repositories.

\subsection{Création de l'utilisateur \lstinline{git} sur le serveur}
Aller Apple Menu  \ra{} System settings \ldots \ra{} Users \& Groups \ra{} \userX{} bouton Add User.

Nom complet \lstinline{guiU // sans importance}

Nom du compte (shortname important !) : \lstinline{git}

Donner mot de passe

Créer l'utilisateur.

Se loger sur l'utilisateur \lstinline{git} et créer le répertoire \lstinline{reposGit}. 
Doner \lstinline{mkdir} et le nom \lstinline{reposGit}. 

\subsection{Créer un groupe \lstinline{gitgroup}}

Aller Apple Menu  \ra{} System settings \ldots \ra{} Users \& Groups \ra{} \userX{} bouton Add Group \lstinline{gitgroup}.

Ensuite donner :

\begin{lstlisting}
sudo chown git /Users/git/reposGit
sudo chgrp -R gitgroup /User/git/reposGit
\end{lstlisting}

Vérifier :

\begin{lstlisting}
ls -ld /Users/git/repoGit
\end{lstlisting}

Résultat attendu :
\begin{lstlisting}
drwxrwsr-x git gitgroup ...  /Users/git/reposGit
\end{lstlisting}

\subsection{Création des utilisateur (développeur) sur le serveur}
Il faut créer des comptes utilisateurs sur le serveur macOS (\serv{}) via l’interface graphique.

Aller Apple Menu  \ra{} System settings \ldots \ra{} Users \& Groups \ra{} \userX{} bouton Add User.

Choisir : Utilisateur standard (recommandé pour accès \lstinline{SSH} + \git{})

Donner :

Nom complet \lstinline{r g r  // sans importance}

Nom du compte (shortname important !), par exemple : \userOne{}

Mot de passe

Créer l'utilisateurs nécessaires.

Aller Apple Menu  \ra{} System settings \ldots \ra{} Users \& Groups \ra{} \userX{} bouton I de  \lstinline{gitgroup} et ajouter les utilisateurs qui vont utiliser \git{}.

\subsection{Vérification sur le serveur que les comptes existent}

\begin{lstlisting}
dscl . -list /Users
\end{lstlisting}


\subsection{Activer "Remote Login" sur serveur}

Pour avoir un accès par un client à l'extérieur du serveur. Sur le serveur configurer:

Aller Apple Menu  \ra{} System settings \ldots \ra{} General \ra{} Sharing Activer Remote Login.

Activer File Sharing, Remote Management, RemoteLoging, et Remote Application Scripting.
À chaque option aller dans i et faites le nécessaire.


\subsection{Vérification  du remote login sur serveur}

Dans le terminal du serveur :

\begin{lstlisting}
sudo systemsetup -getremotelogin
\end{lstlisting}

Doit afficher :

\begin{lstlisting}
Remote Login: On
\end{lstlisting}



\subsection{Vérification qu'un utilisateur sur Mac client  a accès au serveur}
Accéder au serveur depuis un autre Mac
Connexion SSH simple (mot de passe)

Sur le Mac user :

\begin{lstlisting}
ssh rgrmini1@aplixserveur.local
\end{lstlisting}



La première fois, il faut répondre :

\begin{lstlisting}
Are you sure you want to continue connecting (yes/no/[fingerprint])?
\end{lstlisting}

Donner 

\begin{lstlisting}
yes
\end{lstlisting}


%\section{Sécuriser \git +  \lstinline{SSH} }


\begin{figure}[H]
\centering
\includegraphics[scale=0.060]{images/ssh.png}
\caption{ssh}
\end{figure} \label{fig:ssh}


\subsection{Créer sur le client un Clé SSH moderne et forte (ED25519)}

Sur le client. À faire une seule fois . Aller dans \lstinline{~/.ssh} :

\begin{lstlisting}
ssh-keygen -t ed25519 -a 100  -C "id_user1_key"
\end{lstlisting}

Donner le nom du fichier \lstinline{id_user1_key}
pas de passphrase ou choix d'en mettre une.


ED25519 (rapide + sûr)

-a 100 = dérivation lente (anti-bruteforce)

Enter passphrase obligatoire, ne pas choisir empty for no passphrase.

Donner une passphrase, elle sera demande à la connection, plus tard.

Il demande aussi dans quel fichier sauvegarder clés donner : \lstinline{id_user1_key}.
Ceci crée deux fichiers \lstinline{id_user1_key} et \lstinline{id_user1_key.pub}

Le prefix \lstinline{id}  est utilisé dans une autre commande pour copier la clé du client au serveur.

Vérifier la présence des clés sur le client:

\begin{lstlisting}
ls -l ~/.ssh
\end{lstlisting}

Attendu :

\begin{lstlisting}
id_user1_key (cle privee)
id_user1_key.pub (cle publique)
\end{lstlisting}



\subsection{Vérifier les permissions \lstinline{SSH}}

\lstinline{SSH} refuse de fonctionner si les permissions sont mauvaises. 
50 \% des bugs \lstinline{SSH} viennent de là.


\subsection{Permissions STRICTES (indispensable)}

Sur client :

\begin{lstlisting}
chmod 700 ~/.ssh
chmod 600 ~/.ssh/id_user1_key
chmod 644 ~/.ssh/id_useri1_key.pub
chmod 600 ~/.ssh/config
\end{lstlisting}


Sur le serveur, sur le compte client \userOne{}, éventuellement créer le répertoires:

\begin{lstlisting}
chmod 700 ~/.ssh
chmod 600 ~/.ssh/authorized_keys
chmod 700 ~
\end{lstlisting}


\subsection{Copier la clé publique vers le serveur}

Méthode recommandée (simple)

\begin{lstlisting}
ssh-copy-id user1@myserver.local
\end{lstlisting}

Il faut avoir le fichier \lstinline{id_user1_key} le \lstinline{id_} est important.

OU, si ssh-copy-id indisponible sur macOS :

Méthode manuelle (fiable avec les changements de permissions)

\begin{lstlisting}
cat ~/.ssh/id_ed25519.pub | ssh user1@myserver.local\
'mkdir -p ~/.ssh && chmod 700 ~/.ssh && cat >> ~/.ssh/authorized_keys && chmod 600 ~/.ssh/authorized_keys'
\end{lstlisting}


\subsection{Vérifier côté serveur que la clé est bien installée}
Sur le serveur :

\begin{lstlisting}
ls -l ~/.ssh
\end{lstlisting}

Attendu :

\begin{lstlisting}
drwx------  .ssh
-rw-------  authorized_keys
\end{lstlisting}

Vérifier que le fichier contient la clé :

\begin{lstlisting}
cat ~/.ssh/authorized_keys
\end{lstlisting}

Le texte de la clé publique doit se trouver.

\subsection{Test de la connexion \lstinline{SSH}}
Donnez :

\begin{lstlisting}
ssh user1@myserver.local
\end{lstlisting}

Premier message attendu :

\begin{lstlisting}
The authenticity of host 'IP (IP)' can't be established.
Are you sure you want to continue connecting (yes/no)?
\end{lstlisting}

Cela signifie que la connexion fonctionne et que  \lstinline{SSH} n’a pas encore de clé pour ce serveur \ra{}  tapez :

\begin{lstlisting}
yes
\end{lstlisting}

\subsection{Vérifier que l’authentification par clé fonctionne}
Sur le serveur :

\begin{lstlisting}
grep -i PubkeyAuthentication /etc/ssh/sshd_config
\end{lstlisting}

Doit être :

\begin{lstlisting}
PubkeyAuthentication yes
\end{lstlisting}

Et idéalement désactiver password login (optionnel, plus sécurisé) :

Dans \lstinline{/etc/ssh/sshd_config} :

\begin{lstlisting}
PasswordAuthentication no
\end{lstlisting}

Recharger \lstinline{SSH} :

\begin{lstlisting}
sudo launchctl stop com.openssh.sshd 
\end{lstlisting}

\subsection{Débogage : vérifier pourquoi la connexion échoue}
Sur le client :

\begin{lstlisting}
ssh -vvv user1@myserveur.local
\end{lstlisting}

Ceci produit la trace de la commande \lstinline{ssh -vvv user@IP_du_serveur}

Chercher :

\begin{lstlisting}
Offering public key
Authentication succeeded
\end{lstlisting}

Si problème de permissions, vous verrez :

Authentication refused: bad ownership or modes for file ...

Vérifier la connectivité réseau selon \ref{ping}

Vérifier lque le serveur écoute  selon \ref{serverListen}

\subsection{Vérifier que le client a accès au port 22 du serveur}

Vérifier que le port 22 répond :

\begin{lstlisting}
nc -vz IP_du_serveur 22
\end{lstlisting}

Attendu :

\begin{lstlisting}
Connection to IP port 22 [tcp/ssh] succeeded!
\end{lstlisting}


Ta clé privée est protégée par une passphrase. C’est pour ça que tu es invité à la saisir. Cela n’a rien à voir avec le mot de passe du compte \userOne{}.
Si tu veux éviter de taper la passphrase à chaque connexion, tu as deux solutions:

Voici la configuration propre et définitive sur macOS pour ne plus jamais retaper la passphrase, tout en restant sécurisé.

Objectif
La clé reste protégée par passphrase
Tu ne tapes la passphrase qu’une seule fois après ouverture de session
SSH fonctionne ensuite sans interruption





\subsection{Keychain (macOS Apple)}

\begin{lstlisting}
ssh-add --apple-use-keychain ~/.ssh/id_userX_key
\end{lstlisting}

Vérifie :

\begin{lstlisting}
ssh-add -l
\end{lstlisting}

 Si Clé listée \ra{} OK sinon 
 rien\ra{} problème agent


\subsection{Recharger la config}
source ~/.zshrc

\subsection{Vérifier que c’est bien l’OpenSSH Apple}
which ssh
ssh -V

\subsection{Nettoyer l’ancien ssh-agent (optionnel mais recommandé)}
pkill ssh-agent


Puis ouvre un nouveau terminal.

\subsection{Ajouter la clé au Keychain (maintenant ça marche)}


(ou sur macOS très récent)

\begin{lstlisting}
ssh-add --apple-use-keychain ~/.ssh/id_rgrmini1_key
\end{lstlisting}

\subsection{Config finale ~/.ssh/config}

\begin{lstlisting}
Host *
  AddKeysToAgent yes
  UseKeychain yes
  IdentityFile ~/.ssh/id_ed25519
\end{lstlisting}

\subsection{Test final après reboot}
\begin{lstlisting}
ssh-add -l
\end{lstlisting}


 clé listée sans redemander la passphrase 

 Ne fais PAS ça

\begin{itemize}
 \item désinstaller OpenSSH via Homebrew (inutile)
 \item supprimer /usr/local/bin/ssh à la main
\item  relancer \lstinline{eval "$(ssh-agent -s)"}
\end{itemize}









%%%%%%%%%%%%%%%%%%%%%%%%
%%%%%%%%%%%%%%%%%%%%%%%%%%



\subsection{Configuration SSH blindée}

~/.ssh/config :

\begin{lstlisting}
Host *
  IdentityAgent SSH_AUTH_SOCK
  AddKeysToAgent yes
  UseKeychain yes
  IdentitiesOnly yes
  PreferredAuthentications publickey
\end{lstlisting}

Pourquoi c’est important

Option	Rôle

IdentitiesOnly yes	empêche d’essayer d’autres clés

PreferredAuthentications	bloque password / keyboard

IdentityAgent	force l’agent

UseKeychain	persistance après reboot

\subsection{ Une clé par service (recommandé )}
Exemple :

\begin{lstlisting}
ssh-keygen -t ed25519 -f ~/.ssh/id_git_github -C "github"
ssh-keygen -t ed25519 -f ~/.ssh/id_git_gitlab -C "gitlab"
\end{lstlisting}

Config :

\begin{lstlisting}
Host github.com
  HostName github.com
  User git
  IdentityFile ~/.ssh/id_git_github

Host gitlab.com
  HostName gitlab.com
  User git
  IdentityFile ~/.ssh/id_git_gitlab
\end{lstlisting}

 Si une clé fuit → une seule plateforme compromise.

\subsection{Git forcé en SSH (ZÉRO HTTPS)}

\begin{lstlisting}
git config --global url."ssh://git@github.com/".insteadOf "https://github.com/"
git config --global url."ssh://git@gitlab.com/".insteadOf "https://gitlab.com/"
\end{lstlisting}

Vérifie :

git remote -v

\subsection{Vérifier l’empreinte du serveur (anti-MITM)}

\begin{lstlisting}
ssh-keyscan github.com >> ~/.ssh/known_hosts
ssh-keyscan gitlab.com >> ~/.ssh/known_hosts
chmod 644 ~/.ssh/known_hosts
\end{lstlisting}

 Évite les attaques “man-in-the-middle”.

\subsection{Serveur Git personnel (si tu en as un)}

Sur le serveur :

\begin{lstlisting}
chmod 700 ~/.ssh
chmod 600 ~/.ssh/authorized_keys
\end{lstlisting}

DANS \lstinline{authorized_keys }:

\begin{lstlisting}
from="IP_CLIENT",no-port-forwarding,no-X11-forwarding,no-agent-forwarding ssh-ed25519 AAAA...
\end{lstlisting}

Limitation maximale.

\subsection{Désactiver l’authentification par mot de passe (serveur)}

\begin{lstlisting}
/etc/ssh/sshd_config :


PasswordAuthentication no
PermitRootLogin no
PubkeyAuthentication yes
\end{lstlisting}

Puis :

sudo systemctl restart sshd

\subsection{ Tests finaux}
\begin{lstlisting}
ssh -T git@github.com
ssh -v git@github.com
git clone git@github.com:user/repo.git
\end{lstlisting}

 Pas de mot de passe
 Authentification publickey

 Ce qu’il NE FAUT JAMAIS faire

\begin{itemize}
\item  clé sans passphrase
\item  une seule clé partout
\item  \lstinline{eval "$(ssh-agent -s)" sur macOS}
 \item chmod 777 ~/.ssh
\end{itemize}

 Niveau expert (optionnel)

YubiKey / FIDO2 (sk-ssh-ed25519)

ProxyJump + bastion

Match User git dans \lstinline{ssh_config}

git config --global gpg.format ssh

%\section{Le tunnel ssh}
Un tunnel SSH (ou SSH port forwarding) sert à faire passer du trafic réseau à travers une connexion SSH chiffrée.

Il permet :

\begin{itemize}

\item Sécuriser un service non sécurisé

Exemple : accéder en HTTPS via un tunnel à un serveur HTTP local non exposé.

\item Accéder à un service distant comme s’il était local

Exemple : accéder à un serveur SVN, MySQL, ou une interface web située derrière un firewall.

\item Contourner des restrictions réseau

Exemple : accéder à une machine du LAN privé via un serveur bastion.

\end{itemize}

Le principe :

Le client SSH ouvre un port local (ex. : 8080)

Tout ce qui va à localhost:8080 passe dans un tunnel chiffré

Le serveur SSH renvoie le trafic à la machine/port cible

Sécuriser SVN

Le protocole svn:// n’est pas chiffré.
Avec SSH → tout est chiffré automatiquement.

 Accéder à un serveur SVN non exposé

Par exemple, si le serveur SVN écoute en interne sur localhost:3690, sans être ouvert sur Internet.

Contourner un firewall / routeur

Si l’accès direct au port SVN (3690) est bloqué.

 Accéder au repository SVN comme s’il était en local

Le client SVN croit qu’il parle à localhost, mais le tunnel redirige vers le serveur.

\subsection{ Configuration du tunnel SSH pour SVN}

Supposons :

Serveur SVN écoute sur : localhost:3690

Serveur SSH : user@serveur.example.com

Sur ton Mac, on crée un port local : 3690

Commande du tunnel SSH (mode Local Forwarding)
\begin{lstlisting}
ssh -L 3690:localhost:3690 user@serveur.example.com    
\end{lstlisting}



Ce que cela fait :

Quand SVN accède à localhost:3690 sur ton Mac
→ le trafic passe par SSH
→ et arrive sur serveur.example.com:3690

\subsection{Vérification du tunnel SSH avant d’utiliser SVN} 
Étape 1 : Vérifier que le tunnel fonctionne

Dans un autre terminal, après avoir lancé le tunnel :

nc -zv localhost 3690


Tu dois voir :

Connection to localhost port 3690 succeeded!

Étape 2 : Vérifier que SVN répond depuis le tunnel

Toujours dans un terminal :

svn info svn://localhost/


Si cela fonctionne → tu as bien accès au repo via SSH.

\subsection{Utiliser SVN via le tunnel SSH}

Dans ton client SVN, tu peux maintenant utiliser :

\begin{lstlisting}
 svn co svn://localhost/nom_du_repo   
\end{lstlisting}


ou

svn ls svn://localhost/


 Le client parle à localhost, mais en réalité il utilise le port folwardé via SSH.

\subsection{Tunnel SSH persistant pour SVN (facultatif)}

Pour éviter de relancer la commande manuellement :

Installer autossh (macOS) :
brew install autossh

Lancer un tunnel fiable :
autossh -M 0 -L 3690:localhost:3690 user@serveur.example.com

\subsection{Problèmes fréquents (et solutions)}
 Le tunnel ne se connecte pas

Vérifier le port SSH :

ssh user@serveur.example.com

 SVN renvoie une erreur « Connection refused »

Le service SVN n’écoute peut-être que sur IPv4.
Essaye :

ssh -L 3690:127.0.0.1:3690 user@serveur.example.com

 Le serveur SVN n’est pas lancé

Sur le serveur :

\begin{lstlisting}
 ps aux | grep svnserve   
\end{lstlisting}



ou lancer :

svnserve -d -r /chemin/vers/repos

\subsection{Résumé rapide}
Étape	Action
1	Lancer tunnel SSH : ssh -L 3690:localhost:3690 user@serveur
2	Vérifier port : nc -zv localhost 3690
3	Vérifier SVN : svn info svn://localhost/
4	Utiliser SVN : svn co svn://localhost/repo

%\section{Les repos de \git{}}

2 scénarios courants :

\begin{enumerate}
\item  créer un nouveau dépôt Git (repo)
\item  créer un projet qui utilise ce repo (local + serveur)
\end{enumerate}

\subsection{Créer un nouveau dépôt \git{}}

\begin{enumerate}
\item  Repo local (le plus simple)

À utiliser si tu travailles seul ou avant de pousser vers un serveur.

\begin{lstlisting}
mkdir mon-projet
cd mon-projet
git init
\end{lstlisting}

Résultat :

Un dossier \lstinline{.git/} est créé

Ton projet est maintenant un repo \git{}.

Vérifie :

git status

\item Repo serveur (bare) (nécessaire pour partage)

Un repo bare sert uniquement à recevoir des push.

\begin{lstlisting}
mkdir mon-projet.git
cd mon-projet.git
git init --bare
\end{lstlisting}

Important : Pas de fichiers de code ici. Uniquement pour push / pull
\end{enumerate}

\subsection{ Créer un projet qui utilise ce repo}

\begin{enumerate}

\item  Projet local \ra{} push vers serveur
Étape 1 : créer le projet

\begin{lstlisting}
mkdir mon-projet
cd mon-projet
git init
\end{lstlisting}



Ajoute un fichier :

\begin{lstlisting}
echo "# Mon projet" > README.md
git add .
git commit -m "Initial commit"
\end{lstlisting}

\begin{lstlisting}
git branch -m master main
\end{lstlisting}

Étape 2 : lier au repo distant
\begin{lstlisting}
git remote add origin ssh://user@serveur/chemin/mon-projet.git
\end{lstlisting}


Vérifie :
\begin{lstlisting}
git remote -v
\end{lstlisting}

Étape 3 : créer la branche principale et pousser

\begin{lstlisting}
git branch -M main
git push -u origin main
\end{lstlisting}


 Le projet utilise maintenant le repo distant

\item  Projet depuis un repo existant (clone)

\begin{lstlisting}
git clone ssh://user@serveur/chemin/mon-projet.git
cd mon-projet
\end{lstlisting}

\end{enumerate}


Tu peux commencer à coder directement

\subsection{Cycle de travail Git (quotidien)}

\begin{lstlisting}
git status
git add .
git commit -m "Description du changement"
git push
\end{lstlisting}


Pour récupérer les changements :

\begin{lstlisting}
git pull
\end{lstlisting}

\subsection{Cas particulier : repo bare sans branche}

Si après clone tu vois aucune branche :

\begin{lstlisting}
git branch -M main
git push -u origin main
\end{lstlisting}

Ou vérifie :

\begin{lstlisting}
git symbolic-ref HEAD refs/heads/main
\end{lstlisting}

Résumé rapide 
Action	Commande
Créer repo local	git init
Créer repo serveur	git init --bare
Lier repo distant	git remote add origin ...
Premier push	git push -u origin main
Cloner un repo	git clone ...

Si tu veux, je peux ensuite :

te montrer exactement la même chose dans Eclipse

expliquer Git + Gradle

ou faire un schéma visuel du flux Git



%\input{debug}

%\section{Travail avec Git}

\subsection{Administration}

\subsection{Utilisation de Git par clients}

%\section{Conclusion}


\subsection{Utiliser ssh-agent pour mémoriser la clé}

\begin{enumerate}
    \item Démarrer l'agent SSH
\begin{lstlisting}
 eval "$(ssh-agent -s)"   
\end{lstlisting}

\item Ajouter ta clé privée à l'agent

\begin{lstlisting}
ssh-add ~/.ssh/id_rgrmini1_key
\end{lstlisting}
Ensuite, tu peux te connecter normalement :

\begin{lstlisting}
ssh rgrmini1@aplixserver.local
\end{lstlisting}
L’agent fournira la clé sans demander la passphrase à chaque fois.
\end{enumerate}



\end{document}
