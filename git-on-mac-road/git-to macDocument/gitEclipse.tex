\section{Eclipse}

Le guide clair et fiable pour Eclipse (EGit), sans trous, depuis zéro jusqu’au premier push, avec les écrans importants et les erreurs à éviter.

Je distingue 2 cas réels (c’est important dans Eclipse).

PRÉREQUIS (à vérifier une fois)

Dans Eclipse :

EGit installé (par défaut dans Eclipse IDE)

SSH configuré (\lstinline{~/.ssh/id_ed25519})

Test SSH OK :

ssh -T user@serveur


Dans Eclipse :

Window \ra{} Preferences \ra{} General \ra{} Network Connections \ra{} SSH2


Private keys : \lstinline{~/.ssh/id_ed25519}

SSH executable : Native

CAS 1 Créer un NOUVEAU repo Git + projet Eclipse (le plus courant)

Étape 1 — Créer le projet Eclipse
File → New → Project


Exemple Java :

Java → Java Project


Project name : MonProjet

JRE : correct

NE PAS cocher “Create Git repository here”

Clique Finish

 Étape 2 — Partager le projet avec Git
Right click sur le projet
\ra{} Team \ra{} Share Project


Choisir :

Git


Puis :

Use or create repository in parent folder of project


Clique Create Repository
Puis Finish

 Le projet est maintenant un repo Git local

 Étape 3 — Premier commit
Right click projet
\ra{} Team \ra{} Commit


Dans la vue Commit :

Sélectionne les fichiers

Message : Initial commit

 NE PAS cocher “Commit and Push”

Clique Commit

 Étape 4 — Lier au repo distant
Right click projet
\ra{} Team \ra{} Remote \ra{} Push


Dans Push Configuration :

URI :

\begin{lstlisting}
ssh://user@serveur/srv/git/MonProjet.git
\end{lstlisting}

Clique Next

Étape 5 — Branches (IMPORTANT)

Dans Source Ref :

main

Si main n’existe pas :

Right click projet
→ Team → Switch To → New Branch


Branch name : main

Checkout branch : 

Puis recommence le Push.

Étape 6 — Push final

Dans Destination Ref :

refs/heads/main


Clique Finish

 Repo distant opérationnel

CAS 2 Créer un projet Eclipse À PARTIR d’un repo Git existant

Étape 1 — Cloner le repo
File \ra{} Import
\ra{} Git \ra{} Projects from Git
\ra{} Clone URI


URI :

\begin{lstlisting}
ssh://user@serveur/srv/git/MonProjet.git
\end{lstlisting}

Auth :

SSH

User : user

Key : automatique

Étape 2 — Sélection de branches

Toujours cocher au moins une branche

main


Clique Next

Étape 3 — Import du projet

Choisir :

Import existing Eclipse projects


Clique Finish

 Projet prêt

 Cycle de travail Git dans Eclipse
Action	Menu
Commit	Team \ra{} Commit
Push	Team \ra{} Push
Pull	Team \ra{} Pull
Créer branche	Team \ra{} Switch To \ra{} New Branch
Voir statut	Team \ra{} Synchronize
ERREURS FRÉQUENTES \& SOLUTIONS
 “No branches available”

 Le repo bare est vide
Solution :

git push -u origin main

 “Push not permitted”

Permissions serveur incorrectes
 Vérifier :
\begin{lstlisting}
ls -ld /srv/git
ls -ld /srv/git/MonProjet.git

 "detected dubious ownership"
git config --global --add safe.directory /srv/git/MonProjet.git
\end{lstlisting}


RÉSUMÉ EXPRESS
Nouveau projet
New Project \ra{} Share Project \ra{} Commit \ra{} Push

Repo existant
Clone \ra{} Select branch \ra{} Import