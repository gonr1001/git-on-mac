\section{Les repos de \git{}}

2 scénarios courants :

\begin{enumerate}
\item  créer un nouveau dépôt Git (repo)
\item  créer un projet qui utilise ce repo (local + serveur)
\end{enumerate}

\subsection{Créer un nouveau dépôt \git{}}

\begin{enumerate}
\item  Repo local (le plus simple)

À utiliser si tu travailles seul ou avant de pousser vers un serveur.

\begin{lstlisting}
mkdir mon-projet
cd mon-projet
git init
\end{lstlisting}

Résultat :

Un dossier \lstinline{.git/} est créé

Ton projet est maintenant un repo \git{}.

Vérifie :

git status

\item Repo serveur (bare) (nécessaire pour partage)

Un repo bare sert uniquement à recevoir des push.

\begin{lstlisting}
mkdir mon-projet.git
cd mon-projet.git
git init --bare
\end{lstlisting}

Important : Pas de fichiers de code ici. Uniquement pour push / pull
\end{enumerate}

\subsection{ Créer un projet qui utilise ce repo}

\begin{enumerate}

\item  Projet local \ra{} push vers serveur
Étape 1 : créer le projet

\begin{lstlisting}
mkdir mon-projet
cd mon-projet
git init
\end{lstlisting}



Ajoute un fichier :

\begin{lstlisting}
echo "# Mon projet" > README.md
git add .
git commit -m "Initial commit"
\end{lstlisting}

\begin{lstlisting}
git branch -m master main
\end{lstlisting}

Étape 2 : lier au repo distant
\begin{lstlisting}
git remote add origin ssh://git@serveur/chemin/mon-projet.git
\end{lstlisting}


Vérifie :
\begin{lstlisting}
git remote -v
\end{lstlisting}

Étape 3 : créer la branche principale et pousser

\begin{lstlisting}
git branch -M main
git push -u origin main
\end{lstlisting}


 Le projet utilise maintenant le repo distant

\item  Projet depuis un repo existant (clone)

\begin{lstlisting}
git clone ssh://user@serveur/chemin/mon-projet.git
cd mon-projet
\end{lstlisting}

\end{enumerate}


Tu peux commencer à coder directement

\subsection{Cycle de travail Git (quotidien)}

\begin{lstlisting}
git status
git add .
git commit -m "Description du changement"
git push
\end{lstlisting}


Pour récupérer les changements :

\begin{lstlisting}
git pull
\end{lstlisting}

\subsection{Cas particulier : repo bare sans branche}

Si après clone tu vois aucune branche :

\begin{lstlisting}
git branch -M main
git push -u origin main
\end{lstlisting}

Ou vérifie :

\begin{lstlisting}
git symbolic-ref HEAD refs/heads/main
\end{lstlisting}

Résumé rapide 
Action	Commande
Créer repo local	git init
Créer repo serveur	git init --bare
Lier repo distant	git remote add origin ...
Premier push	git push -u origin main
Cloner un repo	git clone ...

Si tu veux, je peux ensuite :

te montrer exactement la même chose dans Eclipse

expliquer Git + Gradle

ou faire un schéma visuel du flux Git


Voici la procédure correcte et complète pour renommer la branche master en main, sans erreur, dans ton contexte (repo bare + SSH + multi-users).

\subsection{Renommer la branche locale}

Sur ta machine cliente :

\begin{lstlisting}
git branch -m master main
\end{lstlisting}

Vérifier :
\begin{lstlisting}
git branch
\end{lstlisting}

tu dois voir  \lstinline{* main}

\subsection{Pousser main vers le serveur}
git push -u origin main

\subsection{Définir main comme branche par défaut (OBLIGATOIRE)}

Sur le serveur, dans le repo bare :

\begin{lstlisting}
cd /Users/git/reposGit/mp.git
git symbolic-ref HEAD refs/heads/main
\end{lstlisting}

Vérifier :

\begin{lstlisting}
cat HEAD
\end{lstlisting}


Résultat attendu :

\begin{lstlisting}
ref: refs/heads/main
\end{lstlisting}

\subsection{Supprimer l’ancienne branche master (optionnel mais propre)}

 uniquement si plus personne ne l’utilise :

\begin{lstlisting}
git push origin --delete master
\end{lstlisting}

\subsection{Mettre à jour les autres clients (si existants)}

Sur chaque machine :

\begin{lstlisting}
git fetch origin
git branch -m master main
git branch -u origin/main main
git remote set-head origin -a
\end{lstlisting}

Vérifications finales

\begin{lstlisting}
git branch -a
git remote show origin
\end{lstlisting}

Tu dois voir :

\begin{lstlisting}
branche locale : main

distante : origin/main

HEAD \ra{} main
\end{lstlisting}


Étape 3 non faite

 Repo bare sans commits

 Fais d’abord un commit :

echo init > README.md
git add README.md
git commit -m "init"
git push origin master


Puis recommence la procédure.




 Résumé ultra-court
 
 \begin{lstlisting}
git branch -m master main
git push -u origin main
cd mp.git && git symbolic-ref HEAD refs/heads/main
git push origin --delete master
\end{lstlisting}



Problème
ssh://rgrmini1@aplixserver/Users/git/reposGit/mp.git




IMPORTANT En multi-user sécurisé, TOUS les accès Git passent par l’utilisateur git
Les utilisateurs sont distingués par leurs clés SSH, pas par le compte système.


Corriger le remote (CÔTÉ CLIENT)

 \begin{lstlisting}
git remote set-url origin ssh://git@aplixserver/Users/git/reposGit/mp.git
\end{lstlisting}

Vérifier :

\begin{lstlisting}
git remote -v
\end{lstlisting}

Résultat attendu :

\begin{lstlisting}
origin  ssh://git@aplixserver/Users/git/reposGit/mp.git (fetch)
origin  ssh://git@aplixserver/Users/git/reposGit/mp.git (push)
\end{lstlisting}

 Vérifier que TA clé SSH est autorisée pour git

Sur le serveur distant :

\begin{lstlisting}
cat /Users/git/.ssh/authorized_keys
\end{lstlisting}

Ta clé publique doit être présente.

 Une clé = un utilisateur réel
Git loguera l’auteur via le commit (user.name, user.email).

Tester la connexion SSH (client)
ssh git@aplixserver


connexion OK → continue
refus → clé absente ou permissions .ssh incorrectes

 Vérifier les droits du repo (serveur)
ls -ld /Users/git/reposGit/mp.git


Attendu :

drwxrwsr-x git gitdev mp.git


Sinon :

sudo chown -R git:gitdev /Users/git/reposGit/mp.git
sudo chmod -R g+ws /Users/git/reposGit/mp.git

\subsection{Push final (client)}
git push -u origin main

Règle d’or en multi-users
Élément	Règle
utilisateur SSH	git
identité dev	clé SSH + commit
répertoires	/Users/git/reposGit/*.git
repo	bare
droits	setgid
