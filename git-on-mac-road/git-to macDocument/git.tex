\section{Les repos de \git{}}

2 scénarios courants :

\begin{enumerate}
\item  créer un nouveau dépôt Git (repo)
\item  créer un projet qui utilise ce repo (local + serveur)
\end{enumerate}

\subsection{Créer un nouveau dépôt \git{}}

\begin{enumerate}
\item  Repo local (le plus simple)

À utiliser si tu travailles seul ou avant de pousser vers un serveur.

\begin{lstlisting}
mkdir mon-projet
cd mon-projet
git init
\end{lstlisting}

Résultat :

Un dossier \lstinline{.git/} est créé

Ton projet est maintenant un repo \git{}.

Vérifie :

git status

\item Repo serveur (bare) (nécessaire pour partage)

Un repo bare sert uniquement à recevoir des push.

\begin{lstlisting}
mkdir mon-projet.git
cd mon-projet.git
git init --bare
\end{lstlisting}

Important : Pas de fichiers de code ici. Uniquement pour push / pull
\end{enumerate}

\subsection{ Créer un projet qui utilise ce repo}

\begin{enumerate}

\item  Projet local \ra{} push vers serveur
Étape 1 : créer le projet

\begin{lstlisting}
mkdir mon-projet
cd mon-projet
git init
\end{lstlisting}



Ajoute un fichier :

\begin{lstlisting}
echo "# Mon projet" > README.md
git add .
git commit -m "Initial commit"
\end{lstlisting}

\begin{lstlisting}
git branch -m master main
\end{lstlisting}

Étape 2 : lier au repo distant
\begin{lstlisting}
git remote add origin ssh://user@serveur/chemin/mon-projet.git
\end{lstlisting}


Vérifie :
\begin{lstlisting}
git remote -v
\end{lstlisting}

Étape 3 : créer la branche principale et pousser

\begin{lstlisting}
git branch -M main
git push -u origin main
\end{lstlisting}


 Le projet utilise maintenant le repo distant

\item  Projet depuis un repo existant (clone)

\begin{lstlisting}
git clone ssh://user@serveur/chemin/mon-projet.git
cd mon-projet
\end{lstlisting}

\end{enumerate}


Tu peux commencer à coder directement

\subsection{Cycle de travail Git (quotidien)}

\begin{lstlisting}
git status
git add .
git commit -m "Description du changement"
git push
\end{lstlisting}


Pour récupérer les changements :

\begin{lstlisting}
git pull
\end{lstlisting}

\subsection{Cas particulier : repo bare sans branche}

Si après clone tu vois aucune branche :

\begin{lstlisting}
git branch -M main
git push -u origin main
\end{lstlisting}

Ou vérifie :

\begin{lstlisting}
git symbolic-ref HEAD refs/heads/main
\end{lstlisting}

Résumé rapide 
Action	Commande
Créer repo local	git init
Créer repo serveur	git init --bare
Lier repo distant	git remote add origin ...
Premier push	git push -u origin main
Cloner un repo	git clone ...

Si tu veux, je peux ensuite :

te montrer exactement la même chose dans Eclipse

expliquer Git + Gradle

ou faire un schéma visuel du flux Git

