\section{Organization du document}
Afin de faciliter la lecture et l'application de ce  document, nous allons utiliser la figure \ref{fig:gitorg} comme référence. 
Le serveur \git{} (bare repository) sera sur la machine \serv{}. Les développeurs seront sur d'autres machines sur le réseau local. Les développeurs seront \userOne{} et 
\userTwo{}, \ldots{}

On suppose que les OS des Macs ont été modifiés selon les applications installés. Donc, il faut faire des vérifications et des adaptations pour pouvoir travailler avec \git{}.

Dans notre cas les machines utilisés ont des processeurs Intel. Lorsqu'il faut installer un logiciel il faut prendre en compte le processeur de la machine.

Il y a plusieurs étapes pour avoir  un  serveur git opérationnel. 

\begin{enumerate}
    \item Accepter des utilisateurs qui vont utiliser \git{} sur le serveur.. 
    \item Avoir un group : \gitG{}. 
    \item L'installation de \git{} dans les machines utilisateurs et sur le serveur.
    \item L'installation de \lstinline{ssh} pour avoir une communication crypté, installer un tunnel \lstinline{ssh} pour éviter d'ouvir le port 22.
    \item L'administration de \git{} et  la gestion de repos et son utilisation.
\end{enumerate}

MacOS est l'interface utilisateur pour le système Unix. Nous avons utilisé en priorité les commandes qui peuvent être données sur MacOS par rapport au commendes Unix.

Files et dirs

Server :
\begin{lstlisting}
.ssh/authorized_keys
.ssh/config

.bash_profile
ssd/

/ etc / ssh / sshd_config

User

.ssh/id_user_key
.ssh/id_user_key.pub
.ssh/config
.ssh/known_hosts
.bash_profile

-- apple - use - keychain
\end{lstlisting}





\section{Pré-conditions à vérifier avant d'avoir des problèmes}
Au niveau de chaque machine : type de processeur. Apple Menu  \ra{} About this Mac \ra{} Processor.

Au niveau de chaque machine s'assurer que l'utilisateur de la machine a un accès admin. Apple Menu  \ra{} System settings \ldots \ra{} Users \& Groups \ra{} \userX{} est admin.

Au niveau de la machine serveur s'assurer du nom \serv{}. Apple Menu  \ra{} System settings \ldots \ra{} Generals \ra{} Local hostname{} est \serv{}.

\subsection{Vérification d'accès au serveur via le réseau sur chaque client} \label{ping}
Rappel : nom du serveur  est \serv{}

\begin{lstlisting}
ping myserveur.local 
\end{lstlisting}

Ou

\begin{lstlisting}
ping adresse_IP
\end{lstlisting}

Réponse reçue
\begin{lstlisting}
64 bytes from 10.0.0.105: icmp_seq=0 ttl=64 timr=38.764 ms
\end{lstlisting}

\subsection{Vérifier que \git{} est installé, sur  un Mac}
Les Mac serveur et développeurs doivent tous avoir \git{}.
Normalement, \git{}  est pré-installé, pour vérifier :

\begin{lstlisting}
git --version 
\end{lstlisting}

Attendu :

\begin{lstlisting}
git version 2.52.0 // ou plus recente
\end{lstlisting}

Ceci est la  version au moment de l'écriture du document.

Pour tous que tous les futurs repos commencent à la branche main

\begin{lstlisting}
git config --global init.defaultBranch main
\end{lstlisting}


\subsection{Vérifier qu’OpenSSH Mac est installé, sur chaque client Macs}

Normalement c'est pré-installé,  mais il est possible que celui de \homebrew{} le soit, pour vérifier :

\begin{lstlisting}
ssh -V
\end{lstlisting}

Attendu :

\begin{lstlisting}
OpenSSH_9.x, LibreSSL ...
\end{lstlisting}

(la version peut être plus basse, c’est normal)

Si ce celui de \homebrew{}, il faut revenir à celui du Mac \lstinline{OpenSSH_9}.
Pour le faire, il faut faire passer \lstinline{/usr/bin} AVANT \homebrew{} dans le PATH

Éditer le \lstinline{ ~/.bash_profile} avec  \lstinline{nano ~/.bash_profile}

Ajouter tout en haut du fichier :

\begin{lstlisting}
# Priority  Apple binaries
export PATH="/usr/bin:/bin:/usr/sbin:/sbin:$PATH"
\end{lstlisting}


Important : avant toute ligne \homebrew{} du type eval \lstinline{"$(/opt/homebrew/bin/brew shellenv)" ou /usr/local/bin}.

Revérifier avec \lstinline{ssh -V} en obtenant \lstinline{OpenSSH_9.x, LibreSSL}.

Pour garder les deux versions :

\begin{lstlisting}
alias ssh-brew="/usr/local/bin/ssh"
alias ssh-apple="/usr/bin/ssh"
\end{lstlisting}


\subsection{Vérification que le  serveur écoute le port 22} \label{serverListen}
Sur le serveur, vérifier que le service \lstinline{SSH} tourne sur le serveur.

À donner sur le serveur pour vérifier du port \lstinline{ssh} qui est normalement le 22:

\begin{lstlisting}
sudo lsof -iTCP:22 -sTCP:LISTEN
\end{lstlisting}
Attendu :

\begin{lstlisting}
sshd    <pid>    root    TCP *:22 (LISTEN)
\end{lstlisting}


Une fois que tout ceci à été effectué et/ou vérifié passer à la suite.