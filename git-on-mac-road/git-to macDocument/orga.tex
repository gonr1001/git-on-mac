\section{Organization}
Afin de faciliter l'installation les noms suivants seront utilisés :
Le serveur \git{} sera sur la machine \serv{}. Les clients seront sur d'autres machines sur le réseau local. Les clients seront \userOne{} et 
\userTwo{}, \ldots{}

On suppose que les OS des Macs ont été modifiés selon les applications installés. Donc, il faut faire des vérifications et des adaptations pour pouvoir travailler avec \git{}.

Dans notre cas les machines utilisés ont des processeurs Intel. Lorsqu'il faut installer un logiciel il faut prendre en compte le processeur de la machine.

Il y a plusieurs étapes pour avoir  un  serveur git opérationnel. 

\begin{enumerate}
    \item Accepter des utilisateurs qui vont utiliser \git{} sur le serveur.. 
    \item Avoir un group : \gitG{}. 
    \item L'installation de \git{} dans les machines utilisateurs et sur le serveur.
    \item L'installation de \lstinline{ssh} pour avoir une communication crypté, installer un tunnel \lstinline{ssh} pour éviter d'ouvir le port 22.
    \item L'administration de \git{} et  gestion de repos et son utilisation.
\end{enumerate}

MacOS est l'interface utilisateur pour le système Unix. Nous avons utilisé les commandes qui peuvent être données sur MacOS par rapport au commendes Unix.

\subsection{Pré-conditions}
Au niveau système :

\begin{longtable} {|l|l|}
\hline Élément	& Requis  \\  \hline  \hline
\endfirsthead

\hline Élément	& Requis  \\   \hline  \hline
\endhead

\hline \multicolumn{1}{|r|}{{Continued on next page}} \\ \hline
\endfoot
 
\hline 
\endlastfoot
OS serveur	& macOS (Intel ou Apple Silicon) \\ \hline
Accès admin	 & Oui \\ \hline
Réseau	Users peuvent  faire & ping au server \\ \hline
DNS ou IP fixe	& Recommandé \\ \hline
Port SSH	& 22 ou personnalisé \\ \hline
\hline
\end{longtable}

\subsection{Vérification d'accès au serveur via le réseau sur chaque client}

\begin{lstlisting}
ping aplixserver.local
\end{lstlisting}

or

\begin{lstlisting}
ping 10.0.0.105
\end{lstlisting}

Réponses reçues
\begin{lstlisting}
64 bytes from 10.0.0.105: icmp_seq=0 ttl=64 timr=38.764 ms
\end{lstlisting}


\subsection{Vérifier qu’OpenSSH est installé, sur le Mac client}

Normalement préinstallé, pour vérifier :

\begin{lstlisting}
ssh -V
\end{lstlisting}

Attendu :

\begin{lstlisting}
OpenSSH_9.x, LibreSSL ...
\end{lstlisting}

(la version peut être plus basse, c’est normal)

Si c'est autre OpenSSH, ce celui de homebrew, doc pour revenir \lstinline{OpenSSH_9}

Il faut faire passer /usr/bin AVANT Homebrew dans le PATH

Édite ton \lstinline{nano ~/.bash_profile}

Ajoute tout en haut du fichier :

\begin{lstlisting}
# Priority  Apple binaries
export PATH="/usr/bin:/bin:/usr/sbin:/sbin:$PATH"
\end{lstlisting}


Important : avant toute ligne Homebrew du type eval \lstinline{"$(/opt/homebrew/bin/brew shellenv)" ou /usr/local/bin}.

Revérifier avec \lstinline{ssh -V}



\subsection{Vérifier que Git est installé, sur le Mac}

Normalement préinstallé, pour vérifier :

\begin{lstlisting}
git --version 
\end{lstlisting}

Attendu :

\begin{lstlisting}
git version 2.52.0 // ou plus recente
\end{lstlisting}

Ceci est la dernière version.


Une fois que tout ceci à été effectué et/ou vérifié passer à la suite.